\chapter{Introduction}

\hspace{0.5cm} 
Matching of cells in a battery is the number one priority for best battery performance. Yet, little attention is paid to ensure that all cells remain matched throughout the battery lifecycle. Normal operations as well as calendar aging cause the battery to mismatch. Permanent or intermittent external short of even a single cell in a multicell battery also causes cell mismatch. In order to ensure safety, a battery management system (BMS) should identify mismatched cells throughout the battery lifecycle. In most cases, cells are rarely monitored for cell matching. Typically, equipment that uses Li-ion batteries has a BMS that only monitors cell voltage $E_{cv}$ and cell surface temperature $T_{surf}$\cite{dey2016sensor}. Some BMS also monitor ampere-hour capacity (Ah-capacity)\cite{giegerich2016open}, and much less often internal impedance $R_s$ at a single frequency \cite{taylor2012system} -\cite{waag2014critical}, typically 1 kHz. In order to fully characterize the impedance, the BMS should be able to accurately measure in-phase and quadrature (i.e., real and imaginary) components over three decades, typically between 1 and 1000 Hz. By monitoring both amplitude and phase shift at multiple frequencies, the BMS will fully characterize the impedance of the anode, cathode, and the electrolyte. To prevent interfering with the equipment that the battery is powering, such a BMS should not add more than a few mV ac voltage between the cell terminals. A BMS with a singlefrequency- impedance-meter, the only type described so far, essentially only monitors $E_{cv}$, $T_{surf}$ , and the electrolyte resistance, $R_s$. A BMS that monitors only the maximum and minimum $E_{cv}$, single frequency impedance, battery’s Ah-capacity, and $T_{surf}$ of a few cells, may not identify mismatched cells. Test data presented below show that monitoring only $E_{cv}$ and $T_{surf}$ does not ensure adequate electrical and thermal safety. In this paper, we demonstrate that multifrequency impedance data (real and quadrature covering three decades of frequency domain) can be correlated reliably to evolving cell mismatches under three common scenarios: cycle life aging, calendar life aging, and overdischarge and overcharge. We also describe a new and practical BMS that in addition to the conventional $E_{cv}$ monitor contains a multifrequency impedance meter. Our impedance-based BMS has small size (10 × 10 cm), low power demand (6 V; 0.75 A dc), and standalone operation (no need for an external processor to manage the battery and direct commands to external controls such as switches and relays).

\section{Test setup}

\hspace{0.5cm} 
Five different types of Li-ion batteries are used in the tests. For all batteries, new cells, rated to 5.3 Ah capacity,  are selected (Swing 5300 model, Boston Power, Boston, MA, USA). The new cells were cycled twice between 2.7 and 4.2 V at $C/4$-rate, brought to 50\% state of charge ($SoC$), rested overnight at room temperature before screening for matching and making the batteries. Only cells that matched within 3 mV, and in multi-frequency impedance within $\pm$0.5\%, and identical in Ah capacity were selected for assembling batteries. Within one day after the cell screening and selection, one three cell battery, one six-cell battery and two individual cells for abuse (over-discharge and overcharge) tests were reserved. Tests on the three-cell battery started within one day after assembly. The six-cell battery and the two spare cells were stored under ambient conditions for six months before testing. The cells were fully ventilated from all sides to prevent unbalanced thermal constrains. The cells were separated by 5 mm air gap from each other. The battery was raised 5 mm above the resting table, and its top was fully open to air to minimize cell-to-cell heat transfer. 

\subsection{Definitions of normal, bad, and matched cells}

\hspace{0.5cm} 
A new cell is normal if 1) its Ah-capacity determined at $C/4$ rate discharge–charge cycle is the same or better than the nameplate capacity and 2) its internal real and quadrature impedances at 1 kHz, 10 Hz, and 1 Hz all match within $\pm$0.5\% of multiple cells from the same lot. An aged cell is normal if 1) its Ah-capacity determined at C/4 rate discharge–charge cycle follows a linear decay that does not exceed 20\% of its nameplate capacity at the manufacturer-specified end of life and 2) its internal real and quadrature impedances at 1 kHz, 10 Hz, and 1 Hz all match within $\pm$5\% of the new normal cell. A cell is bad, within the context of this work, if it has been over-discharged ($E_{cv}$ dropping almost to zero volt) or overcharged ($E_{cv}$ reaching few hundred milli-volts above 4.2 V) at least once. Cells are matched if their voltages are within $\pm$5 mV at full charge, Ah-capacities are within $\pm$5 mAh, and their impedances have a $\pm$0.5\% at each of the following frequencies: 1 kHz, 100 Hz, 10 Hz, and 1 Hz. 


