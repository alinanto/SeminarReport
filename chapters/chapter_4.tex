\chapter{Conclusion}

Conventional BMS monitor only $E_{cv}$ , $T_{surf}$ , and $R_s$ at a single frequency. They are not enough to ensure safety of the Li-ion battery. The tests showed that $E_{cv}$ , $T_{surf}$ and $R_s$ are not indicators for aging or overdischarge-induced cell mismatches. An impedance-based BITS-BMS, described here \cite{8247206}, that measures amplitude and phase shift at multiple frequencies can identify safety-relevant characteristics associated with anode, cathode, and electrolyte malfunction. Traditionally, such multifrequency BMS are large and require high electric power to operate, therefore have not spread. Portable BMS that measure impedance do so at a fixed single frequency. These single-frequency measurements are also limited to measuring either the real or quadrature (imaginary) component. A small-size, low power, standalone BMS that is enabled by multifrequency (1-1000 Hz) impedance meter that has phase and amplitude monitoring capability was shown in this paper. This BITS-BMS enables battery operation within a user-set range (internal temperature, cell voltage, etc.) and communicates with control systems to regulate charge and discharge currents. The BITS-BMS can identify cell mismatch, and ensure thermal safety and electrical efficiency in Li-ion batteries with multiple cells using simultaneous monitoring of $T_{int}$ , $E_{cv}$ and $R_s$ in each cell.